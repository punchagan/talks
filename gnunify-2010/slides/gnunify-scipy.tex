% Created 2010-12-30 Thu 12:42
\documentclass[presentation]{beamer}
\usepackage[utf8]{inputenc}
\usepackage[T1]{fontenc}
\usepackage{fixltx2e}
\usepackage{graphicx}
\usepackage{longtable}
\usepackage{float}
\usepackage{wrapfig}
\usepackage{soul}
\usepackage{textcomp}
\usepackage{marvosym}
\usepackage{wasysym}
\usepackage{latexsym}
\usepackage{amssymb}
\usepackage{hyperref}
\tolerance=1000
\usepackage[english]{babel} \usepackage{ae,aecompl}
\usepackage{mathpazo,courier,euler} \usepackage[scaled=.95]{helvet}
\usepackage{listings}
\lstset{language=Python, basicstyle=\ttfamily\bfseries,
commentstyle=\color{red}\itshape, stringstyle=\color{darkgreen},
showstringspaces=false, keywordstyle=\color{blue}\bfseries}
\providecommand{\alert}[1]{\textbf{#1}}
\usetheme{Copenhagen}\usecolortheme{lily}\useoutertheme{infolines}\setbeamercovered{transparent}
\begin{document}



\title{Intro to SciPy \& Numpy \newline through Image Processing}
\author{Shantanu \& Puneeth}
\date{February, 2010}
\maketitle

\begin{frame}
\frametitle{Outline}
\setcounter{tocdepth}{3}
\tableofcontents
\end{frame}







\section{Introduction}
\label{sec-1}
\begin{frame}
\frametitle{Audience?}
\label{sec-1_1}
\begin{itemize}

\item Basic Knowledge of Python
\label{sec-1_1_1}%
\begin{itemize}
\item data types
\item variables, data-structures
\item looping constructs
\end{itemize}

\item Anybody doing "Scientific" Computation
\label{sec-1_1_2}%
\begin{itemize}
\item Engineering Students, Researchers
\item People using Fortran/C, Matlab/IDL
\end{itemize}
\end{itemize} % ends low level
\end{frame}
\begin{frame}
\frametitle{A Quote}
\label{sec-1_2}

\begin{quote}
In 1998, \ldots{} I came across Python and its numerical extension
(Numeric) while I was looking for ways to analyze large data sets
\ldots{} using a high-level language. I quickly fell in love with Python
programming which is a remarkable statement to make about a
programming language. If I had not seen others with the same view,
I might have seriously doubted my sanity.

-Travis Oliphant, \emph{Numpy Book}
\end{quote}
\end{frame}
\begin{frame}
\frametitle{Checklist}
\label{sec-1_3}


\begin{description}
\item[Installed?]

\begin{itemize}
\item python-numpy
\item python-scipy
\item python-matplotlib
\item ipython
\item python-imaging or pil
\end{itemize}

\item[Files]

\begin{itemize}
\item smoothing.gif
\item unequalized.jpg
\item lena.png
\item image.png
\end{itemize}

\end{description}
\end{frame}
\section{Getting Started}
\label{sec-2}
\begin{frame}[fragile]
\frametitle{Getting Started}
\label{sec-2_1}


\begin{lstlisting}
$ ipython -pylab
\end{lstlisting}


\begin{itemize}
\item Opening an image
\item Showing it
\end{itemize}
\begin{lstlisting}
a = imread('image.png') imshow(a)
\end{lstlisting}
\end{frame}
\begin{frame}
\frametitle{Some attributes}
\label{sec-2_2}

\begin{itemize}
\item \texttt{shape}
\item \texttt{min}, \texttt{max}, \texttt{sum}
\item \texttt{dtype}
\end{itemize}

    \texttt{ipython}?

\begin{itemize}
\item \texttt{array.<Tab>}
\item \texttt{plot?}
\end{itemize}
\end{frame}
\begin{frame}
\frametitle{Basic Operations}
\label{sec-2_3}


\begin{itemize}
\item + - * / ** //
\item Element-wise operations
\end{itemize}
\end{frame}
\begin{frame}
\frametitle{Simple Arrays}
\label{sec-2_4}



\begin{itemize}
\item Straight forward - single dim, multi dim.
\item \texttt{ones}, \texttt{zeros} et. al
\item \texttt{arange}, \texttt{linspace} with shape
\end{itemize}
\end{frame}
\begin{frame}
\frametitle{Accessing (\& Changing) Elements}
\label{sec-2_5}



\begin{itemize}
\item Accessing (\& Changing) individual elements
\item Accessing (\& Changing) Rows
\item Accessing (\& Changing) Columns
\item Accessing in Steps - Striding
\end{itemize}
\end{frame}
\section{Looking at Lena}
\label{sec-3}
\begin{frame}
\frametitle{Chop and Cut Lena!}
\label{sec-3_1}

\begin{itemize}
\item \texttt{a = scipy.lena()}
\item Select regions

\begin{itemize}
\item Top Left Quarter
\item Face Only
\end{itemize}

\item Resize by dropping pixels

\begin{itemize}
\item Alternate pixels
\item 2 in every 3
\end{itemize}

\item RGB channels in colour images

\begin{itemize}
\item \texttt{imread}
\item \texttt{imshow}
\end{itemize}

\end{itemize}
\end{frame}
\begin{frame}
\frametitle{Smoothing Lena}
\label{sec-3_2}
\begin{itemize}

\item A mean filter
\label{sec-3_2_1}%
\begin{itemize}

\item Neighborhoods\\
\label{sec-3_2_1_1}%
\item for loops\\
\label{sec-3_2_1_2}%
\item Array slicing\\
\label{sec-3_2_1_3}%
\item \%run -t (timing it)\\
\label{sec-3_2_1_4}%
\end{itemize} % ends low level

\item A median filter
\label{sec-3_2_2}%
\begin{itemize}

\item for loops - should be easy?\\
\label{sec-3_2_2_1}%
\item Array slicing\\
\label{sec-3_2_2_2}%
\end{itemize} % ends low level

\item ndimage.median\_filter\\
\label{sec-3_2_3}%
\end{itemize} % ends low level
\end{frame}
\begin{frame}
\frametitle{Copies \& Views}
\label{sec-3_3}

\begin{itemize}
\item Slicing and Striding just reference the same memory
\item They produce views of the data, not copies
\end{itemize}
\end{frame}
\section{Histogram Equalization}
\label{sec-4}
\begin{frame}
\frametitle{Obtain Image, Histogram}
\label{sec-4_1}

\begin{itemize}
\item \texttt{imread}
\item \texttt{imshow}

\begin{itemize}
\item normalizes images by default
\end{itemize}

\item \texttt{ndimage.histogram}
\item \texttt{hist}
\item \texttt{cumsum}
\end{itemize}
\end{frame}
\begin{frame}
\frametitle{Useful Plot Commands}
\label{sec-4_2}

\begin{itemize}
\item \texttt{plot}
\item \texttt{figure}
\item \texttt{xlim}, \texttt{ylim}
\item \texttt{savefig}
\end{itemize}
\end{frame}
\begin{frame}
\frametitle{Obtain Normalized Image, Histogram}
\label{sec-4_3}

\begin{itemize}
\item Linear
\item $A = (A-A.min())\frac{255}{A.max()-A.min()}$
\end{itemize}
\end{frame}
\section{Edge detection}
\label{sec-5}
\begin{frame}
\frametitle{Distance}
\label{sec-5_1}

\begin{itemize}
\item A crude algorithm

\begin{itemize}
\item A point is farther than K
\item distance from lower and right neighbor
\end{itemize}

\end{itemize}
\end{frame}
\begin{frame}
\frametitle{Sobel, Prewitt}
\label{sec-5_2}

\begin{itemize}
\item First order algorithms
\item a = [-1, 0, 1], b = [1, 2, 1]; Sobel
\item a = [-1, 0, 1], b = [1, 1, 1]; Prewitt
\end{itemize}
\end{frame}
\section{Looking Ahead}
\label{sec-6}
\begin{frame}
\frametitle{Getting involved}
\label{sec-6_1}

\begin{itemize}
\item Documentation

\begin{itemize}
\item ReStructured Text
\item ``docstrings''
\item modify docstrings without access to source code
\end{itemize}

\item Bug-fixes
      \href{http://www.scipy.org/BugReport}{http://www.scipy.org/BugReport}
\item Testing
\item Code contributions

\begin{itemize}
\item Scikits \href{http://scikits.appspot.com}{http://scikits.appspot.com}
\end{itemize}

\item Web design
\item Community Participation

\begin{itemize}
\item Active on Mailing list
\item Code sprints/Documentation/Bug-fix Days
\end{itemize}

\end{itemize}
\end{frame}
\begin{frame}
\frametitle{References}
\label{sec-6_2}


\begin{itemize}
\item \href{http://docs.python.org/tutorial/index.html}{Python Tutorial}
\item \href{http://www.scipy.org/Tentative_NumPy_Tutorial}{Tentative Numpy Tutorial}
\item \href{http://docs.scipy.org/doc/numpy/reference/}{Numpy Reference Guide}
\item \href{http://docs.scipy.org/doc/scipy/reference/}{Scipy Reference Guide}
\item Wikipedia
\end{itemize}
\end{frame}

\end{document}
