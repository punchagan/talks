% Created 2010-12-29 Wed 11:37
\documentclass[presentation]{beamer}
\usepackage[utf8]{inputenc}
\usepackage[T1]{fontenc}
\usepackage{fixltx2e}
\usepackage{graphicx}
\usepackage{longtable}
\usepackage{float}
\usepackage{wrapfig}
\usepackage{soul}
\usepackage{textcomp}
\usepackage{marvosym}
\usepackage{wasysym}
\usepackage{latexsym}
\usepackage{amssymb}
\usepackage{hyperref}
\tolerance=1000
\usepackage[english]{babel} \usepackage{ae,aecompl}
\usepackage{mathpazo,courier,euler} \usepackage[scaled=.95]{helvet}
\usepackage{listings}
\lstset{language=Python, basicstyle=\ttfamily\bfseries,
commentstyle=\color{red}\itshape, stringstyle=\color{darkgreen},
showstringspaces=false, keywordstyle=\color{blue}\bfseries}
\providecommand{\alert}[1]{\textbf{#1}}
\usetheme{Boadilla}\usecolortheme{seahorse}\useoutertheme{infolines}\setbeamercovered{transparent}
\begin{document}



\title{Songs, Pictures and Python}
\author{Puneeth,  \newline  Chaitanya   \newline  \&  the WWW}
\date{14 December, 2010}
\maketitle









\section*{Intro}
\label{sec-1}
\begin{frame}
\frametitle{The Talk}
\label{sec-1_1}
\begin{block}{Why?}
\label{sec-1_1_1}

\begin{itemize}
\item More participants for Tutorials (\& Sprints)
\item A feeling for the \textbf{power} that Python gives
\item Get students enthused about Python
\end{itemize}
\end{block}
\begin{block}{What?}
\label{sec-1_1_2}

\begin{itemize}
\item Weekend hacks\footnote{\emph{hackers build things, crackers break them} -- ESR }, etc.
\item Trivial \& straight forward -- No ``research''
\item Mostly images
\item Some web crawling and word counting
\end{itemize}
\end{block}
\end{frame}
\section*{Blue!}
\label{sec-2}
\begin{frame}
\frametitle{We \emph{like} to sing what we see!}
\label{sec-2_1}
\begin{block}{Motivation}
\label{sec-2_1_1}

\begin{itemize}
\item ``We sing what we see'' -- Chaitanya
\item Check it out!
\item Lyrics of top 500 songs

\begin{itemize}
\item 5 decades * 100 songs
\item People connect to them
\end{itemize}

\end{itemize}
\end{block}
\begin{block}{How?}
\label{sec-2_1_2}
\begin{itemize}

\item Get the lyrics\\
\label{sec-2_1_2_1}%
\item Count the words\\
\label{sec-2_1_2_2}%
\end{itemize} % ends low level
\end{block}
\end{frame}
\begin{frame}
\frametitle{How \& What?}
\label{sec-2_2}
\begin{block}{Getting the lyrics}
\label{sec-2_2_1}
\begin{itemize}

\item Search for a website\\
\label{sec-2_2_1_1}%
\item Look at html -- very dirty!\\
\label{sec-2_2_1_2}%
\item Simple hard-coded regex\\
\label{sec-2_2_1_3}%
\end{itemize} % ends low level
\end{block}
\begin{block}{Word count}
\label{sec-2_2_2}
\begin{itemize}

\item Very common\\
\label{sec-2_2_2_1}%
\item Already had some code from our tutorials!\\
\label{sec-2_2_2_2}%
\end{itemize} % ends low level
\end{block}
\begin{alertblock}<2->{"Results"?}
\label{sec-2_2_3}


\begin{center}
\begin{tabular}{lrllrllr}
 blue   &  105  &     &  red     &  54  &     &  green   &  30  \\
 black  &   63  &     &  purple  &  33  &     &  yellow  &  11  \\
 brown  &   56  &     &  white   &  32  &     &  pink    &   7  \\
\end{tabular}
\end{center}
\end{alertblock}
\end{frame}
\begin{frame}
\frametitle{Our eyes suck at blue!}
\label{sec-2_3}
\begin{block}{Discussion}
\label{sec-2_3_1}

\begin{itemize}
\item A post on Hacker News.
\item Known facts

\begin{itemize}
\item Luminance vs. Chrominance
\item Sensitivity -- $G > R > B$

\begin{itemize}
\item Bayer filter (Sensor ratios)
\item CIE 1931 $V(\lambda)$, CIE 1978 $V(\lambda)$ (Spectral sensitivity)
\item They try to illustrate this!
\end{itemize}

\end{itemize}

\item Initial plan was to replicate
\item Flaws in their arguments
\end{itemize}
\end{block}
\end{frame}
\begin{frame}
\frametitle{Argument-1}
\label{sec-2_4}

      \begin{center}
        \includegraphics[width=2in]{data/traci.png}
        \includegraphics[width=2in]{data/tracislide.png}  
      \end{center}
\end{frame}
\begin{frame}[fragile]
\frametitle{Argument-1 \ldots{}}
\label{sec-2_5}
\begin{block}{Code}
\label{sec-2_5_1}

\lstset{language=Python}
\begin{lstlisting}
def show_channels(I):
    for i in range(3):
        J = zeros_like(I)
        J[:, :, i] = I[:, :, i]
        figure(i)
        imshow(J)
\end{lstlisting}
\end{block}
\begin{block}{Discussion}
\label{sec-2_5_2}
\begin{itemize}

\item Blue channel is rather dark
\label{sec-2_5_2_1}%
\begin{itemize}

\item intensity of Blue in the image could be less\\
\label{sec-2_5_2_1_1}%
\end{itemize} % ends low level

\item Bayer filter\\
\label{sec-2_5_2_2}%
\end{itemize} % ends low level
\end{block}
\end{frame}
\begin{frame}[fragile]
\frametitle{Argument-1 \ldots{}}
\label{sec-2_6}
\begin{block}{Code}
\label{sec-2_6_1}

\lstset{language=Python}
\begin{lstlisting}
def show_grey_channels(I):
    K = average(I, axis=2)
    for i in range(3):
        J = zeros_like(I)
        J[:, :, i] = K
        figure(i+10)
        imshow(J)
\end{lstlisting}
      
\end{block}
\begin{block}{Discussion}
\label{sec-2_6_2}
\begin{itemize}

\item Get a gray scale image\\
\label{sec-2_6_2_1}%
\item Look at it using R, G, B filters.
\label{sec-2_6_2_2}%
\begin{itemize}

\item Blue and Red still don't look all that sharp\\
\label{sec-2_6_2_2_1}%
\item intensities change, though\\
\label{sec-2_6_2_2_2}%
\end{itemize} % ends low level
\end{itemize} % ends low level
\end{block}
\end{frame}
\begin{frame}
\frametitle{Argument-2}
\label{sec-2_7}

      \begin{center}
        \includegraphics[width=2.5in]{data/traci_matrix.png}
      \end{center}
\end{frame}
\begin{frame}[fragile]
\frametitle{Argument-2 \ldots{}}
\label{sec-2_8}
\begin{block}{Code}
\label{sec-2_8_1}

\lstset{language=Python}
\begin{lstlisting}
def subsample(I):
    for i in range(3):
        J = I.copy()
        J[:, :, i] = zoom(I[::4, ::4, i], 4)
        figure(i)
        imshow(J)
\end{lstlisting}
\lstset{language=Python}
\begin{lstlisting}
def zoom(x, factor=2):
    rows, cols = x.shape
    row_stride, col_stride = x.strides
    view = np.lib.stride_tricks.as_strided(x,
                    (rows, factor, cols, factor),
                    (row_stride, 0, col_stride, 0))
    return view.reshape((rows*factor, cols*factor))
\end{lstlisting}
\end{block}
\end{frame}
\begin{frame}[fragile]
\frametitle{Argument-2 \ldots{}}
\label{sec-2_9}
\begin{block}{Code}
\label{sec-2_9_1}

\lstset{language=Python}
\begin{lstlisting}
def swap_subsample(I, k=1):
    for i in range(3):
        J = zeros_like(I)
        for j in range(3):
            J[:, :, j] = I[:, :, (j+k)%3]
        J[:, :, i] = zoom(I[::4, ::4, (i+k)%3], 4)
        figure(i+10)
        imshow(J)
\end{lstlisting}
      
\end{block}
\begin{block}{Discussion}
\label{sec-2_9_2}
\begin{itemize}

\item We are definitely good with Green!\\
\label{sec-2_9_2_1}%
\item Blue?\\
\label{sec-2_9_2_2}%
\end{itemize} % ends low level
\end{block}
\end{frame}
\begin{frame}
\frametitle{Further}
\label{sec-2_10}
\begin{block}{Explore}
\label{sec-2_10_1}

\begin{itemize}
\item Reducing bit depth rather than pixel width
\item Central vision vs. Peripheral vision
\item Evolutionary aspects
\item Tetrachromancy
\end{itemize}
\end{block}
\end{frame}
\section*{More images}
\label{sec-3}
\begin{frame}
\frametitle{ASCII art}
\label{sec-3_1}
\begin{block}{Very elementary algo}
\label{sec-3_1_1}

\begin{itemize}
\item Convert image to gray-scale
\item Assign intensity to pixel blocks

\begin{itemize}
\item $block\_len:block\_height::char\_len:char\_height$
\end{itemize}

\item Map intensity to visual density of characters
\item Replace block with corresponding character
\end{itemize}

      Works well for \emph{machine generated images}
\end{block}
\end{frame}
\begin{frame}
\frametitle{Further}
\label{sec-3_2}
\begin{block}{Explore}
\label{sec-3_2_1}
\begin{itemize}

\item pre-process images?
\label{sec-3_2_1_1}%
\begin{itemize}

\item for non \emph{machine generated images}\\
\label{sec-3_2_1_1_1}%
\end{itemize} % ends low level

\item shape matching?\\
\label{sec-3_2_1_2}%
\item colourful html\\
\label{sec-3_2_1_3}%
\end{itemize} % ends low level
\end{block}
\begin{block}{Others}
\label{sec-3_2_2}
\begin{itemize}

\item aalib and bb-demo\\
\label{sec-3_2_2_1}%
\item libcaca\\
\label{sec-3_2_2_2}%
\end{itemize} % ends low level
\end{block}
\end{frame}
\begin{frame}
\frametitle{Face Detection}
\label{sec-3_3}
\begin{block}{Motivation}
\label{sec-3_3_1}
\begin{itemize}

\item Exploring Open CV\\
\label{sec-3_3_1_1}%
\item Tutorials have an example on slicing face of Lena\\
\label{sec-3_3_1_2}%
\end{itemize} % ends low level
\end{block}
\begin{block}{\texttt{facedetect.py}}
\label{sec-3_3_2}
\begin{itemize}

\item Uses a \textbf{Haar} Classifier.\\
\label{sec-3_3_2_1}%
\item Apparently, available as a sample in OpenCV\\
\label{sec-3_3_2_2}%
\item Demo with image, camera\\
\label{sec-3_3_2_3}%
\end{itemize} % ends low level
\end{block}
\end{frame}
\section*{Conclusion}
\label{sec-4}
\begin{frame}
\frametitle{I love Python}
\label{sec-4_1}
\begin{block}{Why?}
\label{sec-4_1_1}
\begin{itemize}

\item Lets me focus on the Problem\\
\label{sec-4_1_1_1}%
\item Interactive\\
\label{sec-4_1_1_2}%
\item Readable\\
\label{sec-4_1_1_3}%
\end{itemize} % ends low level
\end{block}
\begin{block}{Travis Oliphant -- Lead Dev of \texttt{numpy}}
\label{sec-4_1_2}

\begin{quote}
In 1998, \ldots{} I came across Python and its numerical extension
(Numeric) while I was looking for ways to analyze large data
sets \ldots{} using a high-level language. I quickly fell in love
with Python programming which is a remarkable statement to make
about a programming language. If I had not seen others with the
same view, I might have seriously doubted my sanity.
\end{quote}
\end{block}
\end{frame}
\begin{frame}
\frametitle{References}
\label{sec-4_2}


\begin{itemize}
\item Human Vision - \href{http://nfggames.com/games/ntsc/visual.shtm}{http://nfggames.com/games/ntsc/visual.shtm}
\item Hacker News - \href{http://news.ycombinator.net/item?id=1891753}{http://news.ycombinator.net/item?id=1891753}
\item Numpy mailing list - \href{http://www.mail-archive.com/numpy-discussion@scipy.org/msg15594.html}{Stefan van der Walt (striding trick)}
\item Active State - \href{http://code.activestate.com/recipes/483756/}{Convert text to image using PIL}
\item OpenCV Documentation
\item Wikipedia
\end{itemize}
      \begin{center}
        \Huge{Thank You!}
      \end{center}
      
      \vspace{.5in}
      
      \begin{tiny}
        \begin{flushright}
          \color{blue}{Created using Emacs Org-mode}
        \end{flushright}
      \end{tiny}
      
\end{frame}

\end{document}
